\chapter{Using ServiceContainer with wsdl2dispatch}
The \WPYDIS{} script generates a service skeleton, typicallly this interface will be subclassed and the 
base class methods will be invoked by the implementation code.  Functionally the methods of the base class will parse the SOAP request into the expected message, which is available by referencing the instance's {\it request} attribute, and return an initialized response message.

\section{running the script}
Running the command below generates the file \emph{WolframSearchService_services_server.py}

\begin{verbatim}
wsdl2dispatch -u http://webservices.wolfram.com/services/SearchServices/WolframSearch2.wsdl
\end{verbatim}

\section{class ServiceInterface}
In the \module{ServiceContainer} infrastructure all generated service skeletons 
subclass \class{ServiceInterface}.  A service instance's \emph{post} attribute 
specifies the path used to contact the service.

\begin{verbatim}
class ServiceInterface:
    '''Defines the interface for use with ServiceContainer Handlers.
    '''

    def __init__(self, post):
        self.post = post

\end{verbatim}
\section{service skeleton}
The skeleton is generated by the \WPYDIS{} script.  The \method{WolframSearch} operation is 
shown below, the request SOAP message is parsed into a python instance and the response 
is initialized and returned.


\begin{verbatim}
# WolframSearchService_services_server.py
class WolframSearchService(ServiceSOAPBinding):
    ...
    def soap_WolframSearch(self, ps):
        self.request = ps.Parse(WolframSearchRequest.typecode)
        return WolframSearchResponse()
    ...
\end{verbatim}


\section{service implementation}
The user must implement the service and dump it into a container that understands
how to correctly dispatch messages to the implementation's various methods.  


\begin{verbatim}
#!/usr/bin/env python 
import sys
from ZSI.ServiceContainer import AsServer
from WolframSearchService_services_server import *

class Service(WolframSearchService):

    def soap_WolframSearch(self, ps):
        rsp = WolframSearchService.soap_WolframSearch(self, ps)
        msg = self.request

        t1 = time.time()
        opts = msg.Options

        rsp.Result = result = rsp.new_Result()
        if opts.Query == 'Newton':
            result.TotalMatches = 1
            result.Matches = match = result.new_Matches()
            item = match.new_Item()
            item.Title = "Fig Newtons"
            item.Score = 10
            item.URL = 'http://www.nabiscoworld.com/newtons/'
            match.Item = [item]

        result.SearchTime = time.time() - t1

        return rsp

if __name__ == "__main__" :
    port = 8080
    AsServer(port, (Service('test'),))

\end{verbatim}

The \function{ZSI.ServiceContainer.AsServer} function is a convenient way to 
start a HTTP server that will dispatch to your various services.  In a
container all services should have unique paths, here the service is
available at "test".  A URL to contact this service at port 8080 would be
\emph{http://localhost:8080/test}

